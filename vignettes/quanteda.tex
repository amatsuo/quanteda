\documentclass[11pt]{article}
\usepackage{amsmath}
\usepackage{pslatex}
\usepackage{natbib}
\usepackage{setspace}
%\usepackage[sc]{mathpazo}
%\usepackage[T1]{fontenc}
\usepackage[margin=1in,pdftex]{geometry}
\setcounter{secnumdepth}{2}
\setcounter{tocdepth}{2}
\usepackage{url}
\usepackage[unicode=true,pdfusetitle,
 bookmarks=true,bookmarksnumbered=true,bookmarksopen=true,bookmarksopenlevel=2,
 breaklinks=false,pdfborder={0 0 1},backref=false,colorlinks=false]{hyperref}
\hypersetup{pdfstartview={XYZ null null 1}}
\usepackage{breakurl}
%\usepackage[authoryear]{natbib}

\newcommand{\quanteda}{\textsf{quanteda}\ }

\usepackage{Sweave}
\begin{document}
\Sconcordance{concordance:quanteda.tex:quanteda.Rnw:%
1 25 1 50 0 1 5 31 1 6 0 7 1 43 0 7 1 19 0 8 1 13 0 7 1 19 0 7 1 13 0 7 %
1}

%\SweaveOpts{concordance=TRUE}




\begin{Schunk}
\begin{Sinput}
> library(knitr)
> # set global chunk options
> opts_chunk$set(fig.path='figures/minimal-', fig.align='center', fig.show='hold')
> options(replace.assign=TRUE,width=80)
\end{Sinput}
\end{Schunk}

\title{Introduction to the Quantitative Analysis of Textual Data Using
  \quanteda\thanks{This research was supported by the European
    Research Council grant ERC-2011-StG 283794-QUANTESS.  Code
    contributors to the project include Alex Herzog, William Lowe, and
    Kohei Watanabe.}}

\author{Kenneth Benoit and Paul Nulty}

\maketitle

\setlength{\parskip}{1ex}
\setlength{\parindent}{0ex}

\section{Introduction: The Rationale for \quanteda}

\quanteda is an R package designed to simplify the process of
quantitative analysis of text from start to finish, making it possible
to turn texts into a structured corpus, conver this corpus into a
quantitative matrix of features extracted from the texts, and to
perform a variety of quanttative analyses on this matrix.  The object
is inference about the data contained in the texts, whether this means
describing characteristics of the texts, inferring quantities of
interests about the texts of their authors, or determining the tone or
topics contained in the texts.  The emphasis of \quanteda is on
\emph{simplicity}: creating a corpus to manage texts and variables
attached to these texts in a straightforward way, and providing
powerful tools to extract features from this corpus that can be
analyzed using quantitative techniques.

The tools for getting texts into a corpus object include: 
\begin{itemize}
\item loading texts from directories of individual files
\item loading texts ``manually'' by inserting them into a corpus using
  helper functions
\item managing text encodings and conversions from source files into
  corpus texts
\item attaching variables to each text that can be used for grouping,
  reorganizing a corpus, or simply recording additional information to
  supplement quantitative analyses with non-textual data
\item recording meta-data about the sources and creation details for
  the corpus.
\end{itemize}

The tools for working with a corpus include:
\begin{itemize}
\item summarizing the corpus in terms of its language units
\item reshaping the corpus into smaller units or more aggregated units
\item adding to or extracting subsets of a corpus
\item resampling texts of the corpus, for example for use in
  non-parametric bootstrapping of the texts \citep[for an example, see][]{lowebenoitPA2013}
  \item Easy extraction and saving, as a new data frame or corpus, key
    words in context (KWIC)
\end{itemize}

For extracting features from a corpus, \quanteda provides the following tools:
\begin{itemize}
\item extraction of word types
\item extraction of word $n$-grams
\item extraction of dictionary entries from user-defined dictionaries
\item feature selection through
  \begin{itemize}
  \item stemming
  \item random selection
  \item document frequency
  \item word frequency
  \item and a variety of options for cleaning word types, such as
    capitalization and rules for handling punctuation.
  \end{itemize}
\end{itemize}

For analyzing the resulting \emph{document-feature} matrix created
when features are abstracted from a corpus, \quanteda provides:
\begin{itemize}
\item scaling models, such as the Poisson scaling model or Wordscores
\item nonparametric visualization, such as correspondence analysis
\item topic models, such as LDA
\item classifiers, such as Naive Bayes or $k$-nearest neighbour
\item sentiment analysis, using dictionaries
\end{itemize}

\quanteda is hardly unique in providing facilities for working with
text -- the excellent \textsf{tm} package already provides many of the
features we have described.  \quanteda is designed to complement those
packages, as well to simplify the implementation of the
text-to-analysis workflow.  \quanteda corpus structures are simpler
objects than in \textsf{tm}, as are the document-feature matrix
objects from \quanteda, compared to the sparse matrix implementation
found in \textsf{tm}.  However, there is no need to choose only one
package, since we provide translator functions from one matrix or
corpus object to the other in \quanteda.

This vignette is designed to introduce you to \quanteda as well as
provide a tutorial overview of its features.

\section{Installing \quanteda}

The code for the \quanteda package currently resides on
\url{http://github/kbenoit/quanteda}.  From an Internet-connected
computer, you can install the package directly using the
\textsf{devtools} package:

