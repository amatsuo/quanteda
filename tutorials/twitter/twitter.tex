
%this stops output over-running the shaded output area in the pdf 


\documentclass{article}
\title{Introduction to text mining Twitter with R}
\author{Paul Nulty}
\usepackage{hyperref}
\usepackage{Sweave}
\begin{document}
\Sconcordance{concordance:twitter.tex:twitter.Rnw:%
1 1 1 1 4 6 1 1 0 19 1 1 2 6 0 2 1 7 0 1 2 13 1}

\maketitle
\clearpage

\section*{List of Acronyms and Definitions}
\begin{itemize}
  \item API: Application Programming Interface --- a specification of how computer programs can communicate with each other
  \item JSON: JavaScript Object Notation --- a human readable format for representing attribute-value pairs.
  \item OAuth: An open standard for authorization. OAuth provides client applications a `secure delegated access' to server resources on behalf of a resource owner.
  \item HTTP: HyperText Transfer Protocol: 
  \item REST: REpresentational State Transfer:
\end{itemize}

\section*{Useful Links}
\url{https://dev.twitter.com/docs/things-every-developer-should-know}

\clearpage


\begin{Schunk}
\begin{Sinput}
> 2+3
\end{Sinput}
\begin{Soutput}
[1] 5
\end{Soutput}
\begin{Sinput}
> library(quanteda)
> tokenize("Just testing sweave/knitr code integration here.")
\end{Sinput}
\begin{Soutput}
[1] "just"        "testing"     "sweaveknitr" "code"       
[5] "integration" "here"       
\end{Soutput}
\end{Schunk}


\section*{The twitter APIs}
Twitter offers two ways for computer programs to post and retrieve data from its service. The twitter REST Search API will receive requests as http commands, and will respond with a JSON object representing a list of tweets. The twitter streaming API allows a computer program to maintain a connection to twitter's service and receive a flow of live public tweets that match the filters or search terms specified.

Both of these APIs require that the developer of the computer program create a twitter account, create a new application, and receive an access token.

\section*{Getting programmatic access to twitter}
\begin{enumerate}
  \item Go to https://dev.twitter.com/ and sign in with a twitter account
  \item Go to My Applications and select "New App". https://apps.twitter.com/
  \item Fill out the form and click 'create'
\end{enumerate}
\end{document}
