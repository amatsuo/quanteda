\documentclass{article}
\usepackage[sc]{mathpazo}
\usepackage[T1]{fontenc}
\usepackage{geometry}
\geometry{verbose,tmargin=2.5cm,bmargin=2.5cm,lmargin=2.5cm,rmargin=2.5cm}
\setcounter{secnumdepth}{2}
\setcounter{tocdepth}{2}
\usepackage{url}
\usepackage[unicode=true,pdfusetitle,
 bookmarks=true,bookmarksnumbered=true,bookmarksopen=true,bookmarksopenlevel=2,
 breaklinks=false,pdfborder={0 0 1},backref=false,colorlinks=false]
 {hyperref}
\hypersetup{
 pdfstartview={XYZ null null 1}}
\usepackage{breakurl}
\usepackage[authoryear]{natbib}
\usepackage{Sweave}
\begin{document}
\Sconcordance{concordance:scaling.tex:scaling.Rnw:%
1 19 1 50 0 1 5 25 1 6 0 7 1 43 0 7 1 18 0 8 1 13 0 7 1 18 0 7 1 13 0 7 %
1}


\begin{Schunk}
\begin{Sinput}
> library(knitr)
> # set global chunk options
> opts_chunk$set(fig.path='figure/minimal-', fig.align='center', fig.show='hold')
> options(replace.assign=TRUE,width=80)
\end{Sinput}
\end{Schunk}


\title{Unsupervised Document Scaling with Quanteda}


\author{Kenneth Benoit and Paul Nulty}

\maketitle

\section*{Loading Documents into Quanteda}

One of the most common tasks 

The quanteda package provides several functions for loading texts from disk into a quanteda corpus. In this example, we will load a corpus from a set of documents in a directory, where each document's attributes are specified in its filename. In this case, the filename contains the variables of interest, separated by underscores, for example:

\texttt{2010\_BUDGET\_03\_Joan\_Burton\_LAB.txt}

Quanteda provides a function to create a corpus from a directory of documents like this. The user needs to provide the path to the directory, the names of the attribute types, and the character which separates the attribute values in the filenames:

\begin{Schunk}
\begin{Sinput}
> library(quanteda)
> dirname <- "~/Dropbox/QUANTESS/corpora/iebudgets/budget_2010/"
> attNames <- c("year", "debate", "number", "firstname", "surname", "party")
> ieBudgets <- corpusFromFilenames(dirname, c("year", "debate", "no", "fname", "speaker", "party"), sep="_")
\end{Sinput}
\begin{Soutput}
[1] "Reading .. /Users/kbenoit/Dropbox/QUANTESS/corpora/iebudgets/budget_2010//2010_BUDGET_01_Brian_Lenihan_FF.txt"
[1] "Reading .. /Users/kbenoit/Dropbox/QUANTESS/corpora/iebudgets/budget_2010//2010_BUDGET_02_Richard_Bruton_FG.txt"
[1] "Reading .. /Users/kbenoit/Dropbox/QUANTESS/corpora/iebudgets/budget_2010//2010_BUDGET_03_Joan_Burton_LAB.txt"
[1] "Reading .. /Users/kbenoit/Dropbox/QUANTESS/corpora/iebudgets/budget_2010//2010_BUDGET_04_Arthur_Morgan_SF.txt"
[1] "Reading .. /Users/kbenoit/Dropbox/QUANTESS/corpora/iebudgets/budget_2010//2010_BUDGET_05_Brian_Cowen_FF.txt"
[1] "Reading .. /Users/kbenoit/Dropbox/QUANTESS/corpora/iebudgets/budget_2010//2010_BUDGET_06_Enda_Kenny_FG.txt"
[1] "Reading .. /Users/kbenoit/Dropbox/QUANTESS/corpora/iebudgets/budget_2010//2010_BUDGET_07_Kieran_ODonnell_FG.txt"
[1] "Reading .. /Users/kbenoit/Dropbox/QUANTESS/corpora/iebudgets/budget_2010//2010_BUDGET_08_Eamon_Gilmore_LAB.txt"
[1] "Reading .. /Users/kbenoit/Dropbox/QUANTESS/corpora/iebudgets/budget_2010//2010_BUDGET_09_Michael_Higgins_LAB.txt"
[1] "Reading .. /Users/kbenoit/Dropbox/QUANTESS/corpora/iebudgets/budget_2010//2010_BUDGET_10_Ruairi_Quinn_LAB.txt"
[1] "Reading .. /Users/kbenoit/Dropbox/QUANTESS/corpora/iebudgets/budget_2010//2010_BUDGET_11_John_Gormley_Green.txt"
[1] "Reading .. /Users/kbenoit/Dropbox/QUANTESS/corpora/iebudgets/budget_2010//2010_BUDGET_12_Eamon_Ryan_Green.txt"
[1] "Reading .. /Users/kbenoit/Dropbox/QUANTESS/corpora/iebudgets/budget_2010//2010_BUDGET_13_Ciaran_Cuffe_Green.txt"
[1] "Reading .. /Users/kbenoit/Dropbox/QUANTESS/corpora/iebudgets/budget_2010//2010_BUDGET_14_Caoimhghin_OCaolain_SF.txt"
\end{Soutput}
\end{Schunk}


This creates a new quanteda corpus object where each text has been associated values for its attribute types extracted from the filename:

\begin{Schunk}
\begin{Sinput}
> summary(ieBudgets)
\end{Sinput}
\begin{Soutput}
Corpus object contains 14 texts.

                                     Texts Types Tokens Sentences year debate
       2010_BUDGET_01_Brian_Lenihan_FF.txt  1649   7720       390 2010 BUDGET
      2010_BUDGET_02_Richard_Bruton_FG.txt   951   4035       222 2010 BUDGET
        2010_BUDGET_03_Joan_Burton_LAB.txt  1473   5711       329 2010 BUDGET
       2010_BUDGET_04_Arthur_Morgan_SF.txt  1455   6432       349 2010 BUDGET
         2010_BUDGET_05_Brian_Cowen_FF.txt  1470   5835       262 2010 BUDGET
          2010_BUDGET_06_Enda_Kenny_FG.txt  1059   3853       161 2010 BUDGET
     2010_BUDGET_07_Kieran_ODonnell_FG.txt   609   2049       141 2010 BUDGET
      2010_BUDGET_08_Eamon_Gilmore_LAB.txt  1088   3767       208 2010 BUDGET
    2010_BUDGET_09_Michael_Higgins_LAB.txt   439   1132        49 2010 BUDGET
       2010_BUDGET_10_Ruairi_Quinn_LAB.txt   413   1177        60 2010 BUDGET
     2010_BUDGET_11_John_Gormley_Green.txt   362    919        49 2010 BUDGET
       2010_BUDGET_12_Eamon_Ryan_Green.txt   482   1513        90 2010 BUDGET
     2010_BUDGET_13_Ciaran_Cuffe_Green.txt   422   1140        48 2010 BUDGET
 2010_BUDGET_14_Caoimhghin_OCaolain_SF.txt  1040   3614       194 2010 BUDGET
 no      fname  speaker party
 14 Caoimhghin OCaolain    SF
 13     Ciaran    Cuffe Green
 12      Eamon     Ryan Green
 11       John  Gormley Green
 10     Ruairi    Quinn   LAB
 09    Michael  Higgins   LAB
 08      Eamon  Gilmore   LAB
 07     Kieran ODonnell    FG
 06       Enda    Kenny    FG
 05      Brian    Cowen    FF
 04     Arthur   Morgan    SF
 03       Joan   Burton   LAB
 02    Richard   Bruton    FG
 01      Brian  Lenihan    FF

Source:  /Users/kbenoit/Dropbox/QUANTESS/quanteda_kenlocal_gh/tutorials/scaling/* on x86_64 by kbenoit.
Created: Tue Apr 29 14:21:35 2014.
Notes:   NA.
\end{Soutput}
\end{Schunk}

In order to perform statistical analysis such as document scaling, we must extract a matrix containing the frequency of each word type from in document. In quanteda, we use the dfm function to produce such a matrix. \footnote{dfm stands for document-feature matrix --- we say `feature' instead of word, as it is sometimes useful to represent documents by features other than their word frequency.}

\begin{Schunk}
\begin{Sinput}
> docMat <- dfm(ieBudgets)
\end{Sinput}
\begin{Soutput}
Creating dfm: ... done. 
\end{Soutput}
\end{Schunk}

We can now score and plot the documents using a statistical scaling technique, for example correspondence analysis \citep{nenadic2007}.

\begin{Schunk}
\begin{Sinput}
> library(ca)
> model <- ca(t(docMat),nd=1)
> dotchart(model$colcoord[order(model$colcoord[,1]),1], labels = model$colnames[order(model$colcoord[,1])])
\end{Sinput}
\end{Schunk}

This plot indicates the position of each of the documents. We can group documents by their attribute values when creating the word-frequency matrix, which allows us to scale according to a particular party or year, for example

\begin{Schunk}
\begin{Sinput}
> partyMat <- dfm(ieBudgets, group="party")
\end{Sinput}
\begin{Soutput}
Creating dfm: ... aggregating by group: party...complete ... done. 
\end{Soutput}
\begin{Sinput}
> partyModel <- ca(t(partyMat),nd=1)
> dotchart(partyModel$colcoord[order(partyModel$colcoord[,1]),1], labels = partyModel$colnames[order(partyModel$colcoord[,1])])
\end{Sinput}
\end{Schunk}
%\bibliographystyle{authordate2}
%\bibliographystyle{plain}
\bibliographystyle{plainnat}
\bibliography{scaling.bib}

\end{document}
